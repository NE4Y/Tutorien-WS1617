\documentclass[11pt]{article}
\input{/Latex/packages.tex}
\input{/Latex/commands.tex}

\title{Übungsaufgaben: Informatik I WS 16/17 \\ Musterlösung}
\author{Steffen Lindner}

\begin{document}
\maketitle

\section{Records}

\begin{enumerate}
	\item Schreibe eine Daten- und Recorddefinition für 3-dimensionale Vektoren (vector).
	
	\tb{Lösung:}
	
	\begin{lstlisting}
; Ein Vektor (vector) besteht aus
; einer x-Koordinate (vector-x)
; einer y-Koordinate (vector-y)
; einer z-Koordinate (vector-z)
(: make-vector (number number number -> vector))
(: vector? (vector -> boolean))
(: vector-x (vector -> number))
(: vector-y (vector -> number))
(: vector-z (vector -> number))
(define-record-procedures vector
  make-vector
  vector?
  (vector-x
   vector-y
   vector-z))	
	\end{lstlisting}

	\item Schreibe eine Prozedur, die die Länge eines Vectors berechnet.
	
	\tb{Lösung:}
	
	\begin{lstlisting}
(: vector-length (vector -> number))
(check-expect (vector-length v1) 1)
(check-within (vector-length v2) (sqrt 14) 0.01)
(define vector-length
  (lambda (v)
    (sqrt
     (+
      (expt (vector-x v) 2)
      (expt (vector-y v) 2)
      (expt (vector-z v) 2)))))
	
	\end{lstlisting}

\end{enumerate}

\section{Prozeduren auf Listen}

\begin{enumerate}
	\item Schreibe eine Prozedur \tb{sorted?}, die überprüft, ob eine Liste sortiert ist. Die Prozedur akzeptiert eine beliebige Liste und einen Vergleichsoperator für eine lexikalische Ordnung.
	
	\begin{center}
		(: sorted? ((list-of \%a) (\%a \%a $\ra$ boolean) $\ra$ boolean)) 
	\end{center}
	
	\tb{Lösung:}
	
	\begin{lstlisting}
(: sorted? ((list-of %a) (%a %a -> boolean) -> boolean))
(define sorted?
  (lambda (xs comp)
    (cond
      ((empty? xs) #t)
      ((empty? (rest xs)) #t)
      (else
       (and (comp (first xs) (first (rest xs)))
           (sorted? (rest xs) comp))))))
	\end{lstlisting}
	
	\item Schreibe eine Prozedur \tb{sumEven}, die zwei Listen von Zahlen akzeptiert und die geraden Zahlen addiert.
	
	
	
	\begin{center}
		(: sumEven ((list-of number) (list-of number) $\ra$ number))	
	\end{center}
	
	\tb{Lösung:}
	
	\begin{lstlisting}
(: sumEven ((list-of number) (list-of number) -> number))
(check-expect (sumEven (list 1 2 3) (list 4 5 6)) 12)
(define sumEven
  (lambda (xs ys)
    (letrec
        ((flatten
         (lambda (xs ys)
           (append xs ys))))
         (fold 0 +
               (filter (lambda (x)
                         (= (modulo x 2) 0))
                       (flatten xs ys))))))
	\end{lstlisting}
\end{enumerate}

\section{Streams}

\begin{enumerate}
	\item Schreibe eine Prozedur \tb{stream-merge}, die zwei Streams akzeptiert und den alternierenden Stream der beiden Streams bildet.
	
	\begin{center}
		(: stream-merge ((stream-of \%a) (stream-of \%a) $\ra$ (stream-of \%a)))
	\end{center}
	
	\tb{Lösung:}
	
	\begin{lstlisting}
(: stream-merge ((stream-of %a) (stream-of %a) -> (stream-of %a)))
(define stream-merge
  (lambda (s1 s2)
    (make-cons (head s1)
               (lambda () (stream-merge s2 (force (tail s1)))))))
	\end{lstlisting}
\end{enumerate}

\section{Higher Order Procedures - H.O.P}

\begin{enumerate}
	\item Programmieren Sie die Funktionen \tb{filter} und \tb{map} nur durch die Verwendung von fold.

\pm 

\tb{Lösung:}

\begin{lstlisting}
(define filter
  (lambda (p xs)
    (fold empty (lambda (a b) (if (p a) (make-pair a b) b)) xs)))

(define map
  (lambda (p xs)
    (fold empty (lambda (a b) (make-pair (p a) b)) xs)))
\end{lstlisting}

\end{enumerate}

\section{Ausdrücke vereinfachen}

Vereinfache folgende Scheme Ausdrücke.

\ba
	\item \begin{lstlisting}
(define compare
  (lambda (t)
	(cond
	  ((> t 0) #t)
	  ((< t 0) #f)
	  ((= t 0) #t))))	
	\end{lstlisting}
	
	\tb{Lösung:}
	
	\begin{lstlisting}
(define compare
  (lambda (t)
    (>= t 0)))
	\end{lstlisting}
	
	\item 
	
	\begin{lstlisting}
(define f
  (lambda (a b c)
    (not (and a 
	  (or a c
	  (or a b))))))
	\end{lstlisting}
	
	\tb{Lösung:}
	
	\begin{lstlisting}
(define f
  (lambda (a b c)
    (not a)))
	\end{lstlisting}
\ea 

\end{document}
